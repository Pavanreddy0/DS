\documentclass[11pt]{article}
\usepackage{latexsym}
\usepackage[lined,boxed,linesnumbered]{algorithm2e}
\usepackage{amsmath}
\usepackage{amssymb}
\usepackage{amsthm}
\usepackage{epsfig}
\usepackage{psfrag}
\usepackage{color}
%\input{rgb}
\newcommand{\handout}[5]{
  \noindent
  \begin{center}
  \framebox{
    \vbox{
      \hbox to 5.78in { {\bf A model solution for an algorithmic problem} }
      \vspace{4mm}
      \hbox to 5.78in { {\Large \hfill #5  \hfill} }
      \vspace{2mm}
      \hbox to 5.78in { {\em #3 \hfill #4} }
    }
  }
  \end{center}
  \vspace*{4mm}
}

\newcommand{\lecture}[4]{\handout{#1}{#2}{#3}{}{#1}}


\newcommand{\TT}[1]{\textsc{#1}}
\newtheorem{theorem}{Theorem}
\newtheorem{corollary}[theorem]{Corollary}
\newtheorem{lemma}[theorem]{Lemma}
\newtheorem{observation}[theorem]{Observation}
\newtheorem{proposition}[theorem]{Proposition}
%\newtheorem{problem}[theorem]{\color{darkred}{\bf Problem}}
\newtheorem{exercise}{\color{DarkBlue}{\bf Exercise}}
%\newtheorem{exercise}[theorem]{\color{DarkBlue}{\bf Exercise}}
\newtheorem{problem}[theorem]{{\bf Problem}}
\newtheorem{definition}[theorem]{Definition}
\newtheorem{claim}[theorem]{Claim}
\newtheorem{fact}[theorem]{Fact}
\newtheorem{assumption}[theorem]{Assumption}

\topmargin -0.5in
\advance \topmargin by -\headheight
\advance \topmargin by -\headsep
\textheight 9.9in
\oddsidemargin 0pt
\evensidemargin \oddsidemargin
\marginparwidth 0.5in
\textwidth 6.5in

\parindent 0in
\parskip 1.5ex

\begin{document}

\lecture{Maximum sum subarray problem}{}{{\bf } }
~\vspace*{-.5cm}
{\bf Problem:} Given an array $A$ storing $n$ numbers, compute the
subarray of $A$ whose sum is maximum among all possible subarrays of $A$.\\


\noindent
{\bf A model Solution:}

%\section{Algorithm for finding Maximum-Sum Sub Array}
Let $S_i$ denote the sum of the largest-sum subarray ending at (and including) 
$A[i]$. Suppose $S_k$ is maximum among all $S_i$'s. If $A[j,..,k]$ be the 
subarray corresponding to $S_k$, then $A[j,..,k]$ is indeed the subarray of 
$A$ achieving maximum sum. So to solve our problem, it suffices if
we compute $S_i$ for each $i$. Let us develop some insight into $S_i$'s.

It follows from definition that $S_0=A[0]$. We now state the following 
Lemma which establishes a relationship between $S_i$ and $S_{i-1}$ for any 
$i\ge 1$.
%
\begin{lemma}
If $S_{i-1}\leq 0 $ then $S_i=A[i]$; otherwise $S_i=S_{i-1}+A[i]$.
\label{Lemma:Si}
\end{lemma}
\vspace*{-.5cm}
\begin{proof}
The key insight to prove the lemma lies in the following relationship 
between the subarrays associated with $S_i$ and $S_{i-1}$.
Notice that the subarray associated with $S_{i-1}$ if appended with $A[i]$
gives us a subarray of $A$ which terminates at $A[i]$.
Hence, $S_i\ge S_{i-1}+A[i]$. 

Now $S_{i-1}<0$ implies that
every subarray terminating at $A[i-1]$ has a negative sum. In other words,
it is futile to append any subarray terminating at $i-1$ with $A[i]$ to
get a better sum. Hence $S_i=A[i]$ in this case.
Likewise, if $S_{i-1}>0$, then by definition, the subarray corresponding to 
$S_{i-1}$ is the subarray of maximum sum terminating at $A[i-1]$. This 
subarray when appended with $A[i]$ will give us a subarray terminating at
$A[i]$ and having maximum sum. This is exactly the subarray corresponding to 
$S_i$. Hence $S_i=S_{i-1}+A[i]$ in this case.
\end{proof}
%
Algorithm 1 computes maximum sum subarray based on the above lemma.
\IncMargin{1em}
\begin{algorithm}[h]
\SetAlgoLined
\KwData{Input will be an array $ A[0..n-1]$. However, the algorithm will use 
additional arrays $S[0..n-1]$ and $L[0..n-1]$ such that $S[i]$ will store $S_i$ 
and $L[i]$ is the index of the leftmost element of the subarray associated 
with $S_i$.}
\KwResult{The sum of the maximum sum subarray along with its leftmost and 
rightmost indices.}
$ S[0]\leftarrow A[0]$\;
$ L[0]\leftarrow 0 $\;
\For{$ i\leftarrow 1 $ \KwTo $ n-1 $}{
 \eIf{$ S[i-1]\leq 0 $}{
  $S[i]\leftarrow A[i] $\;
  $L[i]\leftarrow i$\;
  }{
  $S[i]\leftarrow S[i-1]+A[i] $\;
  $L[i]\leftarrow L[i-1]$\;
 }
}
Scan array $S$ to compute the maximum element of $S$\;
If $S[k]$ is maximum, output $S[k], L[k], k$.  
\caption{Algorithm for finding Maximum-Sum Sub Array}
\end{algorithm}

%
\noindent
{\bf Proving correctness of the algorithm:}\\
We need to prove that Algorithm 1 is correct. It follows
from the discussion preceding Lemma \ref{Lemma:Si} that it suffices
if we can show that Algorithm 1 correctly computes $S_i$'s.

\begin{lemma}
At the end of the ``for'' loop in Algorithm 1, $S[i]$ stores $S_i$ for each $i<n$.
\label{Lemma:correct}
\end{lemma}
\begin{proof}
We give a proof by induction of $i$. In particular, we show that assertion
${\cal A}(i)$, defined below is true for all $i<n$.\\
${\cal A}(i)$ : At the end of $i$th iteration, $S[i]=S_i$. \\
\noindent
The base case $(i=0)$ is trivially true since we explicitly set $S[0]$ 
to $A[0]$ (which is indeed $S_0$) in the first statement of the algorithm. 
Hence ${\cal A}(0)$ holds true.

Let us prove that ${\cal A}(j)$ is true for some $j>0$, assuming, as induction
hypothesis, that ${\cal A}(i)$ is true for all $i<j$.
Consider the beginning of $j$th iteration. Firstly note that by induction
hypothesis, $S[j-1]=S_{j-1}$. Now Lemma \ref{Lemma:Si} states that $S_j$
is $A[j]$ if $S_{j-1}<0$ and $S_{j-1}+A[j]$ otherwise. Since $S[j-1]$ stores
$S_{j-1}$, it follows from the ``if'' statement executed during $j$th iteration
of ``for'' loop that $S[j]$ stores $S_j$.

Hence by principle of mathematical induction, $S[i]$ stores $S_i$ for
each $i<n$.   
\end{proof}

\begin{large}\textbf{Remember:}\end{large}
It may appear that Lemma \ref{Lemma:correct} or the proof of correctness
of Algorithm 1 is unnecessary since Algorithm 1 seems {\em obviously} correct. 
But the proof is needed for each algorithm irrespective of how simple it looks.
This is due to the following reasons. 
\begin{itemize}
\item
A formal proof of correctness (or sometimes analysis) of an algorithm
is important to rectify some errors. For example, in the binary search 
algorithm, it leads to serious problem if you replace the statement 
$Left \leftarrow mid-1$ by $Left\leftarrow mid$. 
\item
In future, you will find many algorithms whose proof of correctness will not
be obvious. However, using mathematical induction, it may turn out to be
easy provided you formulate the assertion carefully. In that spirit, the
simple proof by induction in Lemma \ref{Lemma:correct} will serve as a useful 
tool for you.
\item If you study the proof of Lemma \ref{Lemma:correct} carefully, you will 
realize that the proof by induction given in Lemma \ref{Lemma:correct} is
nothing but the formal way which you will use to convince yourself about the 
correctness  of Algorithm 1. Isn't it ? It is always good to express your 
thoughts explicitly in words when it comes to analysis of an algorithm. 
It may give you a better insight into the algorithm as well.
\end{itemize}

\noindent
{\bf Analysis of Time and Space complexity the Algorithm}\\

Algorithm 1 executes ``for'' loop $n-1$ times, and in each iteration it spends
$O(1)$ time. Thereafter, it spends a total of $O(n)$ time in the ``for'' loop.
Thereafter, it scans array $S$ to find the maximum element. This also takes
$O(n)$ time. So overall time complexity of the algorithm is $O(n)$.
The algorithm uses two additional arrays $S$ and $L$ in addition to
a few variables. Hence, the algorithm uses $O(n)$ extra space.

\textbf{Note:} We can suitably modify  Algorithm 1 so that it uses just 
a few extra variables instead of the extra arrays $S$ and $L$.
In that case this algorithm runs in $O(n)$ time and takes only $O(1)$ extra 
space. Do it as an exercise.\\

\end{document}
